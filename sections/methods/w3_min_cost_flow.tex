\subsection{Minimum-Cost Maximum Flow}

The minimum-cost maximum flow problem extends the maximum flow formulation by associating a cost $w(u,v)$ with each edge~\cite{course:dte3611:2025}. The objective becomes finding a maximum flow that minimises total transportation cost $\sum_{(u,v) \in E} w(u,v) \cdot f(u,v)$.

The \textbf{Bellman-Ford algorithm} computes shortest paths from a single source while tolerating negative edge weights~\cite{book:clrs:2009}. The algorithm performs $|V|-1$ iterations, each relaxing all edges: for edge $(u,v)$, if $d[u] + w(u,v) < d[v]$, update $d[v]$. This exhaustive approach guarantees correctness since any shortest path contains at most $|V|-1$ edges. A subsequent pass detects negative cycles: if any relaxation remains possible, such a cycle exists. Complexity is $O(VE)$.

The \textbf{cycle-canceling algorithm} leverages Bellman-Ford to optimise flow cost. The procedure operates in two phases:

\begin{enumerate}
\item Compute a maximum flow using any max-flow algorithm (e.g., Edmonds-Karp).
\item While a negative-cost cycle exists in the residual graph, augment flow around it.
\end{enumerate}

In the residual graph, edge costs follow a specific convention: forward edges retain their original cost $w(u,v)$, while backward edges carry negated cost $-w(u,v)$. This encoding ensures that redirecting flow through a backward edge effectively ``refunds'' the cost of the original flow. A negative-cost cycle thus represents an opportunity to rearrange existing flow for reduced total cost without altering the flow value.

Each cycle cancellation strictly decreases total cost while preserving flow conservation. The algorithm terminates when no negative cycles remain, yielding an optimal-cost maximum flow. Worst-case complexity is $O(VE^2 \cdot C)$ where $C$ denotes the maximum edge cost, though practical performance depends heavily on network structure.

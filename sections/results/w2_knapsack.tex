\FloatBarrier
\subsection{Dynamic Programming: Knapsack}

\pgfplotstableread{dat/w2/benchmark_knapsack.dat}{\knapsackdata}

\begin{figure}[H]
  \centering
  \begin{tikzpicture}
    \begin{axis}[
      xlabel={Number of items $n$},
      ylabel={Time (ns)},
      ymode=log,
      legend pos=north west,
      legend style={font=\small},
      width=0.85\textwidth,
      height=0.3\textheight,
      grid=major,
      xtick={10, 15, 20, 25},
    ]
      \addplot[red, mark=triangle*, thick, line width=1.2pt] table[x=n, y=bruteforce_ns] {\knapsackdata};
      \addplot[blue, mark=square*, thick, line width=1.2pt] table[x=n, y=dp_ns] {\knapsackdata};
      \legend{Brute force $O(2^n)$, DP $O(n \cdot W)$}
    \end{axis}
  \end{tikzpicture}
  \caption{0/1 Knapsack: brute-force enumeration versus dynamic programming.}
  \label{fig:knapsack_comparison}
\end{figure}

\Cref{fig:knapsack_comparison} illustrates the dramatic performance divergence between brute-force enumeration and dynamic programming for the 0/1 knapsack problem. The brute-force curve exhibits exponential growth characteristic of $O(2^n)$ complexity, while the DP solution remains nearly constant, consistent with $O(n \cdot W)$ pseudo-polynomial behaviour for fixed capacity $W$.

\begin{table}[H]
  \centering
  \caption{Knapsack benchmark: brute force versus DP ($W = 1000$).}
  \label{tab:knapsack_benchmark}
  \begin{tabular}{rrrr}
    \toprule
    $n$ & Brute force (ns) & DP (ns) & Speedup \\
    \midrule
    10 & 19,926 & 871 & $23\times$ \\
    15 & 627,301 & 1,384 & $453\times$ \\
    20 & 50,876,050 & 2,076 & $24{,}508\times$ \\
    25 & 2,148,728,400 & 2,343 & $916{,}848\times$ \\
    \bottomrule
  \end{tabular}
\end{table}

\Cref{tab:knapsack_benchmark} quantifies the speedup achieved by dynamic programming. At $n = 25$, brute-force enumeration requires approximately 2.15 seconds per instance, whereas DP completes in 2.3 microseconds---a speedup exceeding $9 \times 10^5$. This ratio approximately doubles with each additional item, consistent with the theoretical speedup factor $2^n / (n \cdot W)$.

\subsection{Sorting Algorithms}

\pgfplotstableread{dat/w1/benchmark_sorting.dat}{\sortingdata}

\begin{figure}[H]
  \centering
  \begin{tikzpicture}
    \begin{axis}[
      xlabel={Array size $n$},
      ylabel={Time (ns)},
      xmode=log,
      ymode=log,
      legend pos=north west,
      legend style={font=\small},
      width=0.85\textwidth,
      height=0.27\textheight,
      grid=major,
      log basis x={10},
    ]
      \addplot[blue, mark=square*, thick] table[x=size, y=stlSort] {\sortingdata};
      \addplot[red, mark=triangle*, thick] table[x=size, y=countingSort] {\sortingdata};
      \addplot[green!60!black, mark=o, thick] table[x=size, y=radixSort] {\sortingdata};
      \addplot[orange, mark=diamond*, thick] table[x=size, y=AndAlxSort] {\sortingdata};
      \legend{STL sort, Counting sort, Radix sort, Hybrid quicksort}
    \end{axis}
  \end{tikzpicture}
  \caption{Sorting algorithm performance on pre-sorted input.}
  \label{fig:sorting_sorted}
\end{figure}

\Cref{fig:sorting_sorted} presents execution times on pre-sorted input. STL sort and hybrid quicksort exhibit the expected $O(n \log n)$ scaling behaviour. Counting sort demonstrates its theoretical $O(n)$ advantage for bounded-range integers.

\begin{table}[H]
  \centering
  \caption{Sorting performance on reverse-sorted input (milliseconds).}
  \label{tab:sorting_reverse}
  \begin{tabular}{lrrr}
    \toprule
    Algorithm & $n = 10^4$ & $n = 10^5$ & $n = 10^6$ \\
    \midrule
    STL sort         & 0.04 & 0.45 & 5.43 \\
    Counting sort    & 0.02 & 0.19 & 4.77 \\
    Binary sort      & 0.56 & 6.69 & 102.06 \\
    Radix sort       & 0.08 & 1.13 & 17.71 \\
    Hybrid quicksort & 0.03 & 0.39 & 4.85 \\
    \bottomrule
  \end{tabular}
\end{table}

\Cref{tab:sorting_reverse} details performance on reverse-sorted input. Binary sort, despite achieving correct $O(n \log n)$ asymptotic complexity, exhibits constants approximately $20\times$ higher than STL sort. This overhead stems from dynamic memory allocation and poor cache locality inherent to tree-based structures.

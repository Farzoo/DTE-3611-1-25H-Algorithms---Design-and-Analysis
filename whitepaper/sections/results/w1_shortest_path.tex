\FloatBarrier
\subsection{Shortest Path Algorithms}

\begin{table}[H]
  \centering
  \caption{Pathfinding performance with trivial heuristic $h(v) = 1$.}
  \label{tab:pathfinding}
  \begin{tabular}{lrr}
    \toprule
    Algorithm & Time (ns) & CPU time (ns) \\
    \midrule
    Dijkstra & 17,645 & 16,044 \\
    A*       & 19,975 & 18,834 \\
    \bottomrule
  \end{tabular}
\end{table}

Under trivial heuristic conditions (\cref{tab:pathfinding}), A* exhibits slightly higher execution time than Dijkstra. This result is expected: the uninformative heuristic $h(v) = 1$ provides no guidance toward the goal, causing A* to explore essentially the same vertices as Dijkstra while incurring additional overhead for heuristic evaluation. The true advantage of A* manifests only with informative, admissible heuristics on large search spaces.
